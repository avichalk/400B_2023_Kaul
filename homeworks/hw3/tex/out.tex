\documentclass{article} 
 \usepackage[a4paper, total={8in, 8in}]{geometry} 
 \begin{document} 
\begin{tabular}{lrrlrr}
\hline
 Galaxy Name   &   Halo Mass [$10^{12} M_\odot$] &   Disk Mass [$10^{12} M_\odot$] & Bulge Mass [$10^{12} M_\odot$]   &   Total Mass [$10^{12} M_\odot$] &   $f_{bar}$ \\
\hline
 MW            &                           1.975 &                           0.075 & 0.01                             &                            2.06  &       0.041 \\
 M31           &                           1.921 &                           0.12  & 0.019                            &                            2.06  &       0.067 \\
 M33           &                           0.187 &                           0.009 & N/A                              &                            0.196 &       0.046 \\
\hline
\end{tabular}

\begin{tabular}{l}
\hline
 Total Mass of the Local Group [$10^{12} M_\odot$]: 4.316 \\
 $f_{bar}$ of the Local Group: 0.054                      \\
\hline
\end{tabular}

 1. Total Mass of MW vs M31: Ratio of 1.0. Halo Mass dominates.

 2. Stellar Mass of MW vs M31: Ratio of 0.612. As M31 contains more stellar mass, which is the only type of mass that directly contributes to the luminosity of a galaxy, we expect it to be more luminous.

 3. Dark Matter Content of MW vs M31: Ratio of 1.0. Yes - despite the stark difference in stellar matter, they have almost exactly the same dark matter content!

 4. Baryonic Fraction $f_{bar}$ for each galaxy is MW:0.041, M31:0.067, M33:0.046. Our baryonic fractions are universally lower than the provided value of 16\%. MW:0.25625, M31:0.41875, M33:0.2875. This discrepancy might come down to [TODO!]
 \end{document}
