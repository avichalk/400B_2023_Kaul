\documentclass[a4paper]{article}
\usepackage[utf8]{inputenc}
\usepackage{geometry}
\usepackage{xcolor}
\geometry{
	a4paper,
	total={170mm,257mm},
	left=20mm,
	top=20mm,
}
\usepackage{amsmath, amssymb}
\usepackage{graphicx}
\begin{document}
	\begin{enumerate}
		\item {\bf What is a Galaxy?}
		 \begin{enumerate}
				 \item Baryonic Matter (Gas, ISM, Stars) + Dark Matter. 
				 \item Some galaxies have very small amount of (~100) stars! Some galaxies don't have gas! So, what makes a galaxy? 
				 \item {\it "A galaxy is a gravitationally bound set of stars whose properties cannot be explained by a combination of {\bf baryons} and Newton's laws of gravity."} \footnote{Willman and Strader 2012: "Galaxy" Defined}.
				 \item Sizes range from $1kpc-10skpc$, and luminosities anywhere from  $100L_\odot-10^{12}L_\odot$. Amt. of dark matter varies from $10^8M_\odot=10^{13}M_\odot$. Compare this to the baryonic matter to get the "mass to light ratio", which is the ratio of the total mass to the mass that we can see. 
				 \item But what if Newton is wrong? Use {\bf Milgromian Physics} to explain away behaviour of galaxies without dark matter. 
			 \end{enumerate}
		 \item { \bf What are the components of a galaxy?}
			 \begin{enumerate}
				 \item Baryons. 
				 \item There are diff. distributions. E.g. Disk or Spheroid/Elliptical Distributions.
				 \item {\color{violet}C}{\color{cyan}e}{\color{blue}n}{\color{green}t}{\color{yellow}e}{\color{orange}r}{\color{violet}a}{\color{cyan}l}{\color{blue}i}{\color{green}z}{\color{yellow}e}{\color{orange}d}{\color{violet} }{\color{cyan}S}{\color{blue}u}{\color{green}p}{\color{yellow}e}{\color{orange}r}{\color{violet}m}{\color{cyan}a}{\color{blue}s}{\color{green}s}{\color{yellow}i}{\color{orange}v}{\color{violet}e}{\color{cyan} }{\color{blue}B}{\color{green}l}{\color{yellow}a}{\color{orange}c}{\color{violet}k}{\color{cyan} }{\color{blue}H}{\color{green}o}{\color{yellow}l}{\color{orange}e}{\color{violet}s}{\color{cyan}!}
				 \item Dark Matter Halo: A spherical distribution of dark matter surrounding the disk/spheroid, extending to 10x the size of the disk (assuming Newtonian Dynamics).
				 \item Surrounding the galaxy\ldots
					 \begin{itemize}
					 	\item Satellite Galaxies: smaller galaxies orbiting around the larger one. 
						\item Stellar Halo : Globular clusters, stellar streams, etc surrounding the disk.
						\item Gaseous Halo : gas (typically "hot") $10^6K$ (x-rays) to  $10^4K$ (HI).
					 \end{itemize}
				 \item Dust
			 \end{enumerate}
		 \item {\bf How does a galaxy change over time?}
			 \begin{enumerate}
			 	\item They can run out of gas. This leads to them being {\it quenched}, as there is no detectable amount of star formation at that point. However, there may be mechanisms by which galaxies can be resupplied with gas.
				\item Galaxies can collide!
					\begin{itemize}
						\item This can speed up the process of gas turning into stars.
						\item And black holes can grow by eating gas! The black holes of the galaxies can also collide and merge.
					\end{itemize}
				\item The colour of galaxies can change. Quenched galaxies end up with more reddish stars. So, generally, you go from blue->red. 
			 \end{enumerate}
		 \item {\bf What is a galaxy merger?}
			  \begin{itemize}
			 	\item Galaxy Pairs are different from simply Colliding Systems.
				\item The nuclei need to be merged for the galaxies to be fully merged. Yes, this is observationally limited.
				\item {\bf Dynamical Friction} is important.
					\begin{enumerate}
						\item The gravitational wake that forms behind a galaxy is moving through the dark matter halo of another galaxy.
						\item The wake pulls back on the galaxy, causing it to decelerate. 
						\item And this acts as friction. Without this interaction, galaxies would not merge.
							
					\end{enumerate}
				\item {\bf What is cosmology?} 
					\begin{itemize}
						\item How the universe began, evolved, and will eventually end.
						\item 
							
					\end{itemize}
			 \end{itemize}



	\end{enumerate}
\end{document}
