\documentclass[a4paper]{article}
\usepackage[utf8]{inputenc}
\usepackage{geometry}
\usepackage{xcolor}
\geometry{
	a4paper,
	total={170mm,257mm},
	left=20mm,
	top=20mm,
}
\usepackage{amsmath, amssymb}
\usepackage{graphicx}
\begin{document}
	\begin{enumerate}
		\item {\bf What is a Galaxy?}
		 \begin{enumerate}
				 \item Baryonic Matter (Gas, ISM, Stars) + Dark Matter. 
				 \item Some galaxies have very small amount of (~100) stars! Some galaxies don't have gas! So, what makes a galaxy? 
				 \item {\it "A galaxy is a gravitationally bound set of stars whose properties cannot be explained by a combination of {\bf baryons} and Newton's laws of gravity."} \footnote{Willman and Strader 2012: "Galaxy" Defined}.
				 \item Sizes range from $1kpc-10skpc$, and luminosities anywhere from  $100L_\odot-10^{12}L_\odot$. Amt. of dark matter varies from $10^8M_\odot=10^{13}M_\odot$. Compare this to the baryonic matter to get the "mass to light ratio", which is the ratio of the total mass to the mass that we can see. 
				 \item But what if Newton is wrong? Use {\bf Milgromian Physics} to explain away behaviour of galaxies without dark matter. 
			 \end{enumerate}
		 \item { \bf What are the components of a galaxy?}
			 \begin{enumerate}
				 \item Baryons. 
				 \item There are diff. distributions. E.g. Disk or Spheroid/Elliptical Distributions.
				 \item {\color{violet}C}{\color{cyan}e}{\color{blue}n}{\color{green}t}{\color{yellow}e}{\color{orange}r}{\color{violet}a}{\color{cyan}l}{\color{blue}i}{\color{green}z}{\color{yellow}e}{\color{orange}d}{\color{violet} }{\color{cyan}S}{\color{blue}u}{\color{green}p}{\color{yellow}e}{\color{orange}r}{\color{violet}m}{\color{cyan}a}{\color{blue}s}{\color{green}s}{\color{yellow}i}{\color{orange}v}{\color{violet}e}{\color{cyan} }{\color{blue}B}{\color{green}l}{\color{yellow}a}{\color{orange}c}{\color{violet}k}{\color{cyan} }{\color{blue}H}{\color{green}o}{\color{yellow}l}{\color{orange}e}{\color{violet}s}{\color{cyan}!}
				 \item Dark Matter Halo: A spherical distribution of dark matter surrounding the disk/spheroid, extending to 10x the size of the disk (assuming Newtonian Dynamics).
				 \item Surrounding the galaxy\ldots
					 \begin{itemize}
					 	\item Satellite Galaxies: smaller galaxies orbiting around the larger one. 
						\item Stellar Halo : Globular clusters, stellar streams, etc surrounding the disk.
						\item Gaseous Halo : gas (typically "hot") $10^6K$ (x-rays) to  $10^4K$ (HI).
					 \end{itemize}
				 \item Dust
			 \end{enumerate}
		 \item {\bf How does a galaxy change over time?}
			 \begin{enumerate}
			 	\item They can run out of gas. This leads to them being {\it quenched}, as there is no detectable amount of star formation at that point. However, there may be mechanisms by which galaxies can be resupplied with gas.
				\item Galaxies can collide!
					\begin{itemize}
						\item This can speed up the process of gas turning into stars.
						\item And black holes can grow by eating gas! The black holes of the galaxies can also collide and merge.
					\end{itemize}
				\item The colour of galaxies can change. Quenched galaxies end up with more reddish stars. So, generally, you go from blue->red. 
			 \end{enumerate}
		 \item {\bf What is a galaxy merger?}
			  \begin{itemize}
			 	\item Galaxy Pairs are different from simply Colliding Systems.
				\item The nuclei need to be merged for the galaxies to be fully merged. Yes, this is observationally limited.
				\item {\bf Dynamical Friction} is important.
					\begin{enumerate}
						\item The gravitational wake that forms behind a galaxy is moving through the dark matter halo of another galaxy.
						\item The wake pulls back on the galaxy, causing it to decelerate. 
						\item And this acts as friction. Without this interaction, galaxies would not merge.
							
					\end{enumerate}
				\item {\bf What is cosmology?} 
					\begin{itemize}
						\item How the universe began, evolved, and will eventually end.
						\item 
							
					\end{itemize}
			 \end{itemize}
	\end{enumerate}
\section{17/01/2023}
\begin{enumerate}
	\item {\bf Galactic Structure}. 
		\begin{itemize}
			\item {\bf Baryonic Component.}
				The radial profile of the baryonic disk peaks in the centre of the galaxy, and then reduces as one steps out. The way the brightness ($I $) is measured is that you average the intensity over a given radius). The profile is given by \[
				I(R) = I_{0} e^{-\frac{R}{rs}}
				.\] Where $rs$ is the scale length, the radius at which the density drops off at $e^{-1}$ times the initial value.

				For the Milky Way, we can resolve individual stars, so we can give an {\it number density profile for stars}. \[
				n(R) = n_{0}e^{\frac{R}{rs}}
			.\] Where $n_{0} = 0.02 \frac{stars}{pc^{3}}$, and $rs\sim2-3pc$. 

			Looking towards the south galactic pole and classifying stars and their position above the galactic disk by the function $n(z, R)$ \footnote{Reid, Knude}, we find there are way more G, K stars distant from the main disk, and mostly A stars close to the main disk. This marks the delineation between the thin and thick disks. {\bf C.f. figs. 2 and 3 on notepad. }
				\begin{itemize}
					\item {\bf Thin Disk.} Young, metal/alpha rich stars. Mass : $4.5-6*10^{10}M_{\odot}$.
				\item {\bf Thick Disk.} Old, metal/alpha poor stars. Mass : $2-4*10^{9}M_{\odot}$. 
				\item {\bf Boxy "Bulge"} (though in our Milky Way it's not really a bulge\ldots)
				\item {\bf Gas Disk}. Mostly HI, H2, HIII. Mass : $7*10^{9}M_{\odot}$. 
				 \item {\color{violet}C}{\color{cyan}e}{\color{blue}n}{\color{green}t}{\color{yellow}e}{\color{orange}r}{\color{violet}a}{\color{cyan}l}{\color{blue}i}{\color{green}z}{\color{yellow}e}{\color{orange}d}{\color{violet} }{\color{cyan}S}{\color{blue}u}{\color{green}p}{\color{yellow}e}{\color{orange}r}{\color{violet}m}{\color{cyan}a}{\color{blue}s}{\color{green}s}{\color{yellow}i}{\color{orange}v}{\color{violet}e}{\color{cyan} }{\color{blue}B}{\color{green}l}{\color{yellow}a}{\color{orange}c}{\color{violet}k}{\color{cyan} }{\color{blue}H}{\color{green}o}{\color{yellow}l}{\color{orange}e}{\color{violet}}{\color{cyan}!} Mass : $4.1*10^{6}M_{\odot}$. We have a hard time figuring out the exact distance to it, but since we can view stellar orbits around the black hole, we can precisely figure out the black hole mass. 
				 \item {\bf Bar}. The Milky Way Bar was reently confirmed as 3D by GAIA. This is part of the way that gas moves through our MW, and it helps in funnelling gas to the black hole. 
					 \begin{itemize}
					 	\item It's been found that stellar orbits in the presence of a disk are significantly altered. They can make bulges more boxy, x-shaped, etc. 
					 \end{itemize}

				 \item {\bf Satellite Galaxies}. LMC and SMC are our very own satellite galaxies, and they have a huge amount of HI gas streaming off them in a structure called the {\it Magellanic Stream}. This might be how galaxies capture new gas. The SDSS found around 26 satellite galaxies around our MW, some with around 100 stars!
					 Then we found 25 more in the Southern Hemisphere, and now we have 58 galaxies with GAIA. 

				 \item {\bf Circumgalactic Medium} (mostly cold HI).Cool! Is the main gas source of galaxies such as the MW. You might gain more gas from satellite galaxies, perhaps some that's just always there, and maybe some gas that's released from the MW (in a process called galactic recycling). If the gas is lost entirely, it is known as an outflow. 
Mass: $4.6*10^{10}M_{\odot}$. So, much more than is in the MW. 
\item {\bf Stellar Halo}. Mass : $ \sim_{10}^{9}M_{\odot}$. 
				\end{itemize}
				Note that stars of different ages have {\it different kinematics}, thus leading to different kinematics for the thin and thick disks. 
				\begin{itemize}
					\item Younger stars have velocity dispersions $\sigma_{z}\sim 15 \frac{km}{s}$, while older stars have $\sigma_{z}\sim 25-40\frac{km}{s}$, so they move significantly faster!
					\item The theory is that at the start of formation, you have a mostly stable disk, but that gets disrupted by either interactions with other galaxies or the mechanics of star formation, etc. However, we don't know if this is true for all galaxies, and if these "structures" are even truly structures or just random chance. 
					\item In fact, a study \footnote{Rox and Bovy} shows that there is no bimodal distribution as you would except for such a distinct thin/thick disk configuration. 

				\end{itemize}

			\item {\bf Non-Baryonic Matter}. 
				{\bf Dark Matter}. Mass: $ \sim {10}^{12}M_{\odot}$. 
		\end{itemize}
\end{enumerate}
\section{24/01/2023}
\begin{enumerate}
  \item {\bf Our Local Group}.
    \begin{itemize}
      \item Milky Way.
        \begin{enumerate}
          \item $\approx 30 satellites$
        \end{enumerate}
      \item Andromeda ($\approx$ 770kpc away)
        \begin{enumerate}
          \item $\approx$ 32 satellites
        \end{enumerate}
    \end{itemize}
\end{enumerate}
They are close enough that there is considerable overlap between the two!
\begin{enumerate}
  \item {\bf Proper Motion} of an object is the change in position of an object on the sky when accounting for the rotation of the earth, solar system, etc.
  \item We measured the proper motion of Andromeda, and found that it's moving basically directly towards us.
  \item It will hit us in like 4Gyr.
  \item Let's discuss galactic collisions.
    \begin{itemize}
      \item We need to solve the N-body problem.
      \item But we don't want to model every single star in the galaxy. How do we choose?
      \item We choose "star particles" with masses of $\sim 1000M_{\odot}$, which are taken to represent clumps of stars.
    \end{itemize}
Now, let's discuss rotation curves.
\begin{itemize}
  \item {\bf Rotation Curves}. The variation in the orbital velocity of stars or gas as a function of distance from the centre of the galaxy.
    \begin{gather}
       GMm/R^{2}=mV^{2}/R \\
       M(<R)\propto V^{2}R \\
       v_{c}(R) = \sqrt{GM(<R)/R} \\
    %   v_{c)(R)=\sqrt{GM(<R)/R} \\
    v_{c}(R)=R\omega
    \end{gather}
    Where $\omega$ is the angular velocity.

  \item Let's define a few different types of rotation.
    \begin{enumerate}
      \item {\bf Solid Body}. Angular speed is constant as a function of radius. \[
      V=R\omega, V\propto R
      .\] This will tell us some things about the mass enclosed.\[
      M(<R)\propto V^{2}R\propto R^{3}
      .\] This also tell us about the density.\[
      M(<R)=\text{Volume}*\text{Density}\propto R^{3}\rho
      .\] Where $\rho$ is a constant.
    \item {\bf Keplerian}. Mass is concentrated in the centre.\[
        V\propto\sqrt{M/R}\propto R^{-1/2}
    .\] So, if we start to see a rapid decrease in the velocity at a certain distance, we can assume most of the mass is concentrated at lesser radii.
  \item {\bf Flat Rotation Curve}. V=constant across all radii.\[
  M(<R)\propto V^{2}R \propto R
  .\] So, the mass is increasing linearly with radius. \[
  M \propto R^{3}\rho
  .\] And these two equations give us \[
  \rho \propto R^{-2}
  .\] 
    \end{enumerate}
\end{itemize}
\item {\bf VlSR} : Velocity of the Local Standard of Rest. This is the average velocity of stars near the sun with respect to the galactic centre. The sun will have some pecular motion with respect to the VLSR. We can use this to account for our own net motion in measuring rotation curves of other galaxies.
  We find our own velocity to be $V_{\odot}=VLSR+V_{peculiar}$. We can then find thevelocity of galactic clouds that are along an angle $l$ from our line of sight. C.f. fig. 4 and corresponding math.
  We can use the proper motion of Sag A* to find the velocity of gas clouds at different distances out from the galactic centre.
\end{enumerate}
\section{30/01/2023}
\begin{enumerate}
  \item {\bf Schechter Function}. The log of the number of galaxies vs. the absolute magnitude or mass.
    \begin{itemize}
      \item {\bf Luminosity Function}. $\phi(L)dL=\frac{\text{No. of stars or galaxies between a luminosity} L \text{and} L+dL}{\text{Volume Probed}}$. 

        To find the number density of galaxies, just $\int_{L}^{\infty}\phi(L')dL'$. 
    \end{itemize}
\end{enumerate}
\end{document}
